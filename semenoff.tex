\documentclass[12pt,a4paper]{article}
\usepackage{graphics}
\usepackage{csvsimple}
\usepackage[russian,english]{babel}
\usepackage[utf8]{inputenc}
\usepackage[T2A]{fontenc} 
\hoffset -1.6cm 
\textwidth  16.5cm 
\textheight 24cm 
%\topmargin -1cm 
\parskip 8pt plus 1pt minus 1pt 
\setlength{\unitlength}{1cm}
\sloppy
\addto\captionsenglish{
\renewcommand{\contentsname}{{\bf CO}ntents}
\renewcommand{\refname}{Bibliography}
\renewcommand{\figurename}{Figure}
\renewcommand{\tablename}{Table}
\renewcommand{\abstractname}{Abstract}
\renewcommand{\partname}{Section}
\renewcommand{\bottomfraction}{0.5}
\renewcommand{\floatpagefraction}{0.4}
\renewcommand{\textfloatsep}{0.5cm}
\renewcommand{\intextsep}{0.6cm}
\renewcommand{\floatsep}{0.3cm}
}

\begin{document}
%..................................................................
\begin{titlepage}
\par 
\vspace*{-2cm}
\begin{center}
{\sf \Large
\vspace*{1.5cm}
{\Huge Семенов, 8-й семестр, 2017}\\
{ Все, что вы хотели {\small(или не хотели, но сдавать надо)} знать о философии,\\ но боялись спросить}}\\
\vspace*{2cm}
\scalebox{1.2}{\includegraphics{thegod.png}} 
\begin{flushright}
\sl\small
by Nyaxx11k aka KingKO,\\
Institute for Porn and Pederasts,\\
Chair of Roosterology
\end{flushright}
\end{center}
\end{titlepage}
%..................................................................
\topmargin -1cm 
\hoffset -0.7in 
\textwidth 6.0in 
\textheight 9.0in 
\normalsize 
\pagenumbering{arabic}
%----------------
\tableofcontents
\pagebreak
%----------------

\section{Истина как цель научного исследования. Истина и заблуждение.}
Согласно т.н. "Классическому определению истины", \textbf{истина} - это то, что согласуется с действительностью. 
Соответственно \textbf{заблуждение} - то, что с действительностью не согласуется. \textit{Не ложь - это отрицание истины исключительно в матлогике. В остальных науках это - умышленное введение в заблуждение.}
Ясно, что все науки занимаются поиском истины.
Каждая из них ищет истину о чем-то своем: биология - о живых организмах, физика - о предельно общих законах природы и т.д.

\section{Истина как объект философского исследования.}
Ясно, что все науки занимаются поиском истины.
Каждая из них ищет истину о чем-то своем: биология - о живых организмах, физика - о предельно общих законах природы и т.д.
\textbf{Философия} же ищет истину о самой истине: как достичь истины, не свалившись в заблуждение.

\section{Философия как теория познания и самый общий метод мышления.}
Философия ищет истину о самой истине: как достичь истины, не свалившись в заблуждение.
То есть философия является теорией познания (\textbf{гносеологией}/\textbf{эпистемологией}).
По другому задача философии может быть сформулирована так: как мыслить правильно, т.е. так, чтобы прийти именно к истине.
А это означает, что философия дает наиболее общий метод мышления. 

\section{Понятие объекта и субъекта, объективного и субъективного}
\textbf{Субъект} - это существо, обладающее сознанием и волей, и способное к целенаправленной деятельности,
которую оно направляет на \textbf{объект} - некий предмет/явление.
У нас в курсе деятельность - это познание.
Т.е. субъект познает объект.
\textit{Да, субъект может быть и объектом. Семенов грозился просить привести примеры объектов.
Да, называем все, что видим, и будет счастье.} 
Соответственно, вводят понятия \textbf{субъективного} - оно означает, что что-то зависит от субъекта,
и \textbf{объективного} - того, что от объекта не зависит.

\section{Ступени человеческого познания}
У человека есть два способа познания - \textbf{чувственное познание} и \textbf{мышление} (ака \textbf{умственное познание}).
Первое осуществляется органами чувств и человеку неподвластно.
А второе - это человеческая деятельность, она подвластна сознанию, а значит, и методологию для нее можно разрабатывать (более того, это-то и есть философия как метод мышления).

\section{Чувственное познание. Его основные формы.}
Чувственное познание существует в трех формах, или, скорее, стадиях:
\begin{enumerate}
\item \textbf{Ощущение} - начальная фаза. Объект воздействует на субъекта, и его органы чувств (всего их 5) передают информацию об этом в мозг.
\item \textbf{Восприятие} - на этой ступени данные со всех органов собираются в единый образ предмета.
\item \textbf{Представление} - образ ранее воспринятого предмета может быть воспроизведен в сознании субъекта и в отсутствии этого предмета.
\end{enumerate}
Это познание есть не только у человека, но и у любых животных. Понятно, что чувственного познания не достаточно для того, чтобы познать закономерности мира.

\section{Мышление как деятельность человека. Проблема правильного образа (метода) познания. Философия как метод мышления и наука о мышлении - логика}
\textbf{Мышление} - это уже человеческая деятельность, она подвластна сознанию, а значит, и методологию для нее можно разрабатывать, в отличие от чувственного.
Более того, этим-то и занимается философия.
Значит, философия является еще и наукой о мышлении - логикой.
\textit{Вообще, \textbf{логика} - это наука
о мышлении, в центре внимания которой - его правильность или истинность.\nocite{Sem01} Далее мы увидим, что не все - философия, что логика.
}
\textit{ Следует оговориться, что его она изучает как способ достижения истины, т.е. не интересуется, например, отклонениями в мышлении (это к психиатрам)}
\textit{ При мышлении органы чувств не используются непосредственно. }

\section{Два вида мышления: рассудочное и разумное, и две логики: формальная и содержательная (философская)}
Мышление может быть как субъективной деятельностью - \textbf{рассудочным мышлением},
так и объективным процессом - \textbf{разумным мышлением}. 
У каждого из них логика своя - соответственно, \textbf{формальная} и \textbf{содержательная}.
Формальная логика изучает лишь правильность мышления, а посему от философии с проблемами истины она давно отпочковалась, став самостоятельной наукой. Логика же разумного мышления в философии используется интенсивно. 
\textit{ Это не значит, что философия забивает на чувственное познание и рассудок, и уж тем более, что она их отрицает. Очевидно, для работы разума нужны как чувственные данные, так и рассудок - чтобы стать известными, результаты работы разума в любом случае нужно перевести в формы, присущие рассудку.}

\section{Классическое определение истины}
Без лишних предисловий: \textbf{Истина} - это то, что согласуется с действительностью (\textit{by Платон/Аристотель-у первого оно встречается раньше, но приписывают обычно второму}). 
Т.е. иначе говоря, истина - соответствие между миром и сознанием.

\section{Проблема определения понятий "мир" и "сознание"}
Обычно дают определение через род и видовое отличие. Мир и сознание - предельно общие понятия, еще более общее только одно - бытие (т.е. и мир, и сознание - есть): но это нам ничего не даст - бытие и так все включает. Значит, придется раскрывать отношение - что из них первично, что вторично. Это - \textbf{"основной вопрос философии"}. Именно по ответу на него классифицируют направления философии.

\section{Основной вопрос философии}
Обычно дают определение через род и видовое отличие. Мир и сознание - предельно общие понятия, еще более общее только одно - бытие (т.е. и мир, и сознание - есть): но это нам ничего не даст - бытие и так все включает. Значит, придется раскрывать отношение - что из них первично, что вторично. Это - \textbf{"основной вопрос философии"}. Именно по ответу на него классифицируют направления философии. \textit{Кстати, философ, считающий, что нет никакого основного вопроса - недофилософ и петух. Считающий, что на вопрос нет и не будет ответа - это уже другое.}

\section{Философия как мировоззрение (онтология). Натурфилософия. Социальная философия.}
Итак, определить мир и сознание можно только одним образом - раскрыв отношение между ними. Это значит, что тем самым будут определены оба понятия. Тем самым философия дает еще и предельно общий взгляд на мир, т.е. является \textbf{онтологией}. Раньше, когда науки еще были в зачаточном состоянии, только философия была готова дать хоть какую-нибудь картину мира. Такое учение называется \textbf{натурфилософией}. В настоящее время натурфилософия отмерла за ненадобностью. \textit{Но не онтология - она по прежнему занимается проблемами мира - теми, которые нужны для теории познания}. Но кроме природы есть еще общество - им занимаются \textbf{социальная философия} и \textbf{философия истории}. И с ними все не так однозначно, как с натурфилософией. Во-первых, есть \textbf{общественное сознание} - в широком смысле это все знание человечества; более того, сознание отдельного человека формируется в обществе \textit{(а дети-Маугли ни говорить, ни социализироваться не могут)}. Во-вторых, разные школы придерживаются разных взглядов на общество: первые говорят, что общество - только название, а на самом деле есть только взаимодействующие люди. Вторые таки признают общество. Это два направления - \textbf{социальный номинализм} и \textbf{социальный реализм} соответственно.

\section{Материализм как онтология и как гносеология. Понятие материи.}
Итак, первое направление в философии \textit{(кстати, именно его и придерживается Семенов)} - это \textbf{материализм}: первичен мир, сознание вторично. Т.о. материализм признает объективную реальность - \textbf{материю}.

\section{В чем заключается вторичность сознания по отношению к материи.}
Нельзя сказать, что материалисты объявляют, что сознания нет вообще. Просто сознание не образует отдельного мира, оно зависимо от материи и не существует без нее. Действительно, сознание - продукт мозга, который есть ни что иное, как материя. Но это не самое главное. А самое главное в том, что сознание - \textsl{отражение} материи.

\section{Формы материализма. Натурматериализм и панматериализм.}
Материалисты до Маркса были непоследовательны: в природе они были материалисты (\textbf{натурматериалисты}) при переходе к обществу им приходилось переходить на позиции идеализма. Действительно, казалось бы, общество - это отражение сознания людей. 
\textit{Еще бы, общество в отличие от физических тел не пощупаешь.}
\textit{Не говоря уже о том, что общество в отличие от мира не вечно - оно возникло вместе с людьми.} Маркс же нашел социальную материю - производственные отношения. Тем самым материализм стал полным -\textbf{панматериализмом}, или \textbf{диалектическим материализмом}. Другое важное его следствие - человек не просто пассивный наблюдатель, он воздействует на мир, преобразуя его.


\newpage

\end{document}