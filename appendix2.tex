\section{Аппендиксъ 2: Философский бестиариум <WIP>}
{\scriptsize
\begin{tabular}{|c|c|c|}
\hline
\thead{Кто\\Когда и Где\\Ачивки}&\thead{Философия}&\thead{Кроме философии}

\phylentry{Фалес}
{($\approx$624 д.н.э - $\approx$546 д.н.э.)}
{Милет, Малая Азия\\\
(Гонял по заграницам)}
{-Мир вечен\\\
-Первовещество-вода}
{-Смотался в Египет, там ботал преднауку\\\
-Первый доказал теорему\\\
-Предсказывал солнечные затмения}{
\materialist
}

\phylentry{Анаксимандр}
{($\approx$619 д.н.э - $\approx$546 д.н.э.)}
{Милет, Малая Азия}
{-Мир вечен\\\
-Первовещество-апейрон\\\
-Мир-цилиндр\\\
-Человек-потомок морских животных}
{-Смотался в Египет, там ботал преднауку\\\
-Первый доказал теорему\\\
-Предсказывал солнечные затмения}{}


\phylentry{Анаксимен}
{($\approx$585 д.н.э - $\approx$525 д.н.э.)}
{Милет, Малая Азия}
{-Мир вечен\\\
-Первовещество-воздух\\\
-Мир-диск}
{ВОИДЪ}{}

\phylentry{Гераклит}
{($\approx$540 д.н.э - $\approx$480 д.н.э.)}
{Эфес, Малая Азия}
{-Мир вечен\\\
-Первовещество-огонь\\\
-Диалектика: мир-вечный процесс\\\
-Его двигатель - противоположности и противоречия}
{ВОИДЪ}{}

\phylentry{Пифагор}
{($\approx$570 д.н.э - $\approx$487 д.н.э.)}
{Остров Самос, Эгейское море\\ (Прямо у побережья Турции)\\\
Кротон, Южная Италия\\\
Метапонт, Южная Италия
}{-Мир вечен\\\
-Первовещество-огонь\\\
-Диалектика: мир-вечный процесс\\\
-Его двигатель - противоположности и противоречия}
{ВОИДЪ}{}

\phylentry{Ксенофан}
{($\approx$580 д.н.э - $\approx$490 д.н.э.)}
{Колофон, Малая Азия \\\
\textit{<Бродяжничал>}\\\
Элея, Южная Италия
}{
-Пантеист\\
-Греческая мифология-фантазия}
{-Бродячий певец}{}

\phylentry{Парменид}
{($\approx$546 д.н.э - $\approx$470 д.н.э.)}
{
Элея, Южная Италия
}{
-Учитель Зенона\\\
-Отрицал небытие\\\
-Бытие-вечно, неделимо и неподвижно\\\
-Разделил чувственное и умственное познание\\\
-Т.к. его учение чувствами непостижимо\\\
}
{-Герой книг Платона\\\
-\textit{Педераст - любил юношей младых}}{}


\phylentry{Зенон Элейский}
{($\approx$490 д.н.э - $\approx$430 д.н.э.)}
{
Элея, Южная Италия
}{
-Апориями доказывал Парменида (всего 40, дошло 9)
}
{ВОИДЪ}{}

\phylentry{Мелисс}
{($\approx$490 д.н.э - $\approx$430 д.н.э.)}
{
Остров Самос\\\
\textit{<Эмигрировал>}\\\
Элея, Южная Италия
}{
-Мир вечен\\\
-Мир бесконечен
}
{ВОИДЪ}{}


\\\hline
\end{tabular}
\begin{tabular}{|c|c|c|}
\hline
\thead{Кто\\Когда и Где}&\thead{Философия}&\thead{Кроме философии}

\phylentry{Эмпедокл}
{($\approx$492 д.н.э - $\approx$430 д.н.э.)}
{
Акрагант, Сицилия\\\
}{
-4 стихии ака корни\\\
-Дружба и вражда как движущие силы\\\
-4 фазы развития, зацикленные\\\
-Стихи бесконечно делимы\\\
-Частички проникают через поры\\\
\ \ \ воздействуя на органы чувств.
}
{
-Теория эволюции через рандомную \\ \ \ сборку органов\\\
-Конечная скорость света.
}{}

\phylentry{Анаксагор}
{($\approx$496 д.н.э - $\approx$428 д.н.э.)}
{
Клазомены, Малая Азия\\\
\textit{<Эмигрировал>}\\\
Афины, Греция\\\
\textit{<Грохнуть хотели>}\\\
Лампсак, Малая Азия
}{
-Гомомерии ака семена мира (их дофига)\\\
-Ум (нус) как движущая сила
}
{ВОИДЪ}{}

\phylentry{Левкипп}
{(5-й век д.н.э)}
{
Абдеры, возможно Милет
}{
-Учитель Демокрита\\\
-Единственное, что известно - был
}
{ВОИДЪ}{}


\phylentry{Демокрит}
{($\approx$460 д.н.э - $\approx$360 д.н.э.)}
{
Абдеры, Фракия
}{
-Атомизм: атомы в пустоте (пространстве)\\
-Атомы незримы, неделимы\\
-Чувства - кажимость, есть атомы\\
-Детерминизм: случайность \\ \ \ - от незнания причин\\
-Познание темное(чувства) и светлое(ум)\\
-Делил на микро- и макромир\\
-Миров великое множество
}
{-По легенде, выколол глаза, ибо уже не нужны\\}{}

\phylentry{Протагор}
{($\approx$486 д.н.э - $\approx$411 д.н.э.)}
{
Абдеры, Фракия
}{
-Человек-мера всех вещей\\
-Истина-в полезности, т.е. для каждого своя\\
-Мир один, но воспринимаем по-разному\\
-Естественная и человеческая стадии развития
}
{-Софиствовал\\}{}

\phylentry{Горгий}
{($\approx$480 д.н.э - $\approx$380 д.н.э.)}
{
Леонтины, Италия\\
Афины, Греция\\
Лариса, Греция
}{
-Ничего нет\\
-А если и есть, то непознаваемо\\
-А если и познаваемо, то в словах невыразимо
}
{-Софиствовал\\}{}

\phylentry{Сократ}
{($\approx$469 д.н.э - 399 д.н.э.)}
{
Афины, Греция
}{
-Единство души и тела\\
-Душа содержит знания\\
-Зло от незнания\\
-Ограничить потребности для независимости
}
{-Майевтика\\
-Книг не писал\\
-Выпил йаду за развращение \\\ \ \ молодежи и богохульство\\
-Я знаю, что ничего не знаю\\
-А другие не знают и этого\\
-Глав. добродетели - умеренность, \\\ \ \ мужество, справедливость
}{}

\phylentry{Платон (Аристокл)}
{(427/8 д.н.э - 346/7/8 д.н.э.)}
{
Афины, Греция\\
\textit{поскитался}\\
Остров, Сицилия\\
Афины, Греция\\
Остров, Сицилия\\\textit{хотел вернуться, неудачно}\\
Афины, Греция\\
}{
-Классификация политических строев\\
-Первый утопист - гос-во философов\\
-Родоначальник эстетики\\
-В 376 г д.н.э. создал Академию\\
}
{
-Первый завершенный объективный идеалист\\
-Мир эйдосов-идей\\
-Высшая идея-Благо\\
-Материя вечна, вещи-нет\\
-Есть Бог, творящий, руководствуясь эйдосами\\
-Душа-вечна, тело-нет\\
-3 Компонента души-Чувственная,\\ \ \  Пылкая,Разумная\\
-Душа припоминает мир эйдосов \\ отсюда у нас общее и врожденные понятия\\
-Рассказ "Пещера"; эйдосы там-реальный мир
}{}

\\\hline
\end{tabular}
\begin{tabular}{|c|c|c|}
\hline
\thead{Кто\\Когда и Где}&\thead{Философия}&\thead{Кроме философии}

\phylentry{Аристотель}
{(384 д.н.э - 322 д.н.э.)}
{
Стагир, Фракия(Македония?)\\\
Афины, Греция\\\
Халкида, Остров Эвбея
}{
-Учение о бытии\\
-10 Категорий\\
-Вещи-из материи\\
-Материя пассивна\\
-Вещь=Материя+Форма\\
-Высшая форма-Бог\\
-Новый вещи-в результате движения\\
-Двигает Бог\\
-Где движение, там и энергия\\
-6 форм движения\\
-4 причины образования вещей\\
-Выделил законы формальной логики\\
-Индукция/дедукция. Под конец изучал\\ \ \  только последнюю\\
-Изучал силлогизмы
}
{
-Воспитывал Искандера ака Александра Македонского\\
-Расклассифицировал строи\\
-3 ячейки общества - семья, селение, полис\\
-Мысли о классовой борьбе\\
-55 небесных сфер, снизу 4 стихии, сверху эфир.
}{}

\phylentry{Антисфен}
{ ($\approx$435 д.н.э.-$\approx$370 д.н.э.)}
{
Афины\\
}{
-Киник (Циник)\\
-\textit{Первый номиналист}\\
-Аскетизм ради независимости
}
{
ВОИДЪ
}{}

\phylentry{Диоген}
{ ($\approx$390 д.н.э.-323 д.н.э.)}
{
Синоп, Турция\\
Афины, Греция\\
Коринф, Греция
}{
-Киник
}
{
-Жил в бочке (точнее, в глиняном кувшине)\\
-Встретился с Александром Македонским;\\ \ Просил не загораживать Солнце\\
-Опровергал Платона петухом\\
-Дрочил посреди главной площади
}{}

\phylentry{Эпикур}
{ ($\approx$341 д.н.э.-$\approx$279 д.н.э.)}
{
Остров Самос, Турция\\
Колофон, Малая Азия\\
Мелитен, Остров Лесбос\\
Лампсак, Турция\\
Афины, Греция
}{
-Делил знание по восходящей на\\ логику, физику, этику\\
-Атомист, НО\\
-Органически вплел случайность в атомизм:\\ \ иначе все бы просто падало\\
-Благо - наслаждение\\
-Но не гедонист - умеренность нужна\\
-Высшее благо-беседа с умными людьми\\
-\textit{Пантеист}
}
{
-Добродетели - апатия, бесстрашие, невозмутимость\\
-\textit{Человек-результат эволюции}
}{}

\phylentry{Тит Лукреций Кар}
{ ($\approx$99 д.н.э.-55 д.н.э.)}
{
Помпеи, Италия\\
Рим, Италия
}{
-Неизвестно, что у него свое,\\ а что-Эпикура
}
{
-Доксограф Эпикура\\
-Написал книгу "О природе вещей"
}{}

\phylentry{Зенон Китийский}
{($\approx$334 д.н.э.-$\approx$262 д.н.э.)}
{
Китион, Кипр\\
Афины, Греция
}{
-Создатель стоицизма\\
-Логика/физика/этика как яйцо\\
-Учение об адиафоре-\\безразличных для достижения счастья вещах\\
-Детерменизм\\
-Свободная воля как \\возможность подчиниться\\
-Философствовали в портике,\\ отсюда и название
}
{
-Был купцом, пока корабль не утонул близ Афин
}{}

\phylentry{Крисипп}
{ ($\approx$280 д.н.э.- $\approx$205 д.н.э.)}
{
Солы, Киликия (над Сирией)\\
Афины, Греция
}{
-Скептик\\
-3 вопроса\\
-Только кажимость\\
-Допускаем все и не паримся
}
{
ВОИДЪ
}{}

\\\hline
\end{tabular}
\begin{tabular}{|c|c|c|}
\hline
\thead{Кто\\Когда и Где}&\thead{Философия}&\thead{Кроме философии}


\phylentry{Пиррон}
{ ($\approx$365 д.н.э.- $\approx$275 д.н.э.)}
{
Элида, Пелопоннес, Греция\\
Афины, Греция
}{
-То же, что и у Зенона Китийского
}
{
ВОИДЪ
}{}

\phylentry{Тимон}
{ ($\approx$320 д.н.э.- $\approx$230 д.н.э.)}
{
Флиунт, Пелопоннес, Греция\\
Афины, Греция
}{
-То же, что и у Пиррона
}
{
-\textit{Сатирические стихи писал}
}{}

\phylentry{Марк Туллий Цицерон}
{ ($\approx$106 д.н.э.- $\approx$46 д.н.э.)}
{
Рим, Италия (?)\\
}{
-Эклектик
}
{
-Оратор
}{}


\phylentry{Луций Анний Сенека }
{ (6 д.н.э.-65 н.э.)}
{
Рим, Италия (?)\\
}{
-Считал, что знание не гарантирует добродетель\\
-Богатство юзать с умом\\
-Материя+дух\\
-Дух=божество+судьба
}
{
-Юрист, администратор\\
-Был в оппозиции\\
-Воспитал Нерона\\
-41-49 гг провел в ссылке\\
-Обвинен в заговоре\\
-Вскрылся по приказу Нерона
}{}



\phylentry{Эпиктет}
{ (55 н.э.-130 н.э.)}
{
Гиераполь, Фригия\\
Никополь, Эпир
}{
-Беседовал\\
-Стоик\\
-Раб-господин - все относительно\\
-Адиафора
}
{
-Вольноотпущенный раб
}{}

\phylentry{Марк Аврелий Антонин}
{ (121 н.э.-180 н.э.)}
{
Гиераполь, Фригия\\
Никополь, Эпир
}{
-Жизнь коротка и бренна\\
-Смирение\\
}
{
-Император (последний из пяти хороших )
}{}


\\\hline
\end{tabular}
}

Тертуллиан (155 н.э.-225 н.э.)\\
Карфагенец. Критик античной философии - язычники же все. Выразил концепцию Троицы. Верует, ибо абсурдно. Создал \textbf{патристику} - ХГМ-направление в философии.

Климент Александрийский (150 н.э.-220 н.э.)\\
Считал, что христианство можно доказать, используя достижения античной философии.

Ориген (185 н.э.-254 н.э.)\\
Те же взгляды, что и у предыдущего. Еретик, ибо ставил Бога-Отца выше Сына.

Аммоний Саккас (175 - 242)\\
Александриец. Основатель неоплатонизма.

Плотин (206 - 275)\\
Ученик предыдущего. У него есть понятие единого первоначала. Из него получаются божественный ум, божественная душа и природа.

Порфирий (204 - 301)

Ямвлих (245 - 330)\\
Какое-то богословие и только

Прокл (410 - 485)

Августин Аврелий Блаженный (354 - 430)
Создал первую философию истории.Она конечто, далека от реальности, но все же: Бог наметил план истории, и история подстраивается под него - \textbf{провиденционализм}. Есть Град земной - в нем живут по плоти, и Град божий - там живут по духовным ценностям (\textit{с христанутостью это не скоррелировано!}). На Страшном Суде их пихают в Ад и Рай соответственно. Государство он приравнивал к банде. Концепция у него унитарная - человечество движется как единое целое. Но прогресса нет, а будет только конец света.

Северин Боэций (480-524)\\
Был приговорен к казни за участие в заговоре, в тюрьме написал книгу "Утешение философией", в которой занимал позиции стоиков. Утверждал, что Бог знает будущее, но это не отменяет свободной воли индивида. Переводил труды.

Аврелий Кассиодор ($\approx$485-$\approx$585)\\
Создатель концепции "Семи свободных искусств". Они делись на две ступени - \textbf{тривиум} и \textbf{квадрилиум}.

Иоанн Скот Эриугена (810-877)\\
Ирландец. Жил и работал при дворе Карла Великого. С примесью  богословия, но все-таки философ. Наследник патристики. Ставил знание выше веры. То, что все знание содержится в Библии, означает, что его нужно уметь подчерпнуть оттуда. Считал, что общее существует - был \textbf{реалистом}. Считал, что есть три формы существования общего: в божественном разуме, в вещах, в голове человека.

Ансельм Кентерберийский Д'Аоста. (1033-1109)\\
Глава английской церкви. Веру ставил выше знания: да, без знания никуда, но оно служит вере. Придумал \textbf{онтологическое доказательство} существования Бога: "Бог - совершенное существо. Но как может совершенное существо не существовать" (\textit{софист долбаный!}). \textbf{Реалист} - считал, что общее существует  - это план, по которому Бог создавал этот мир; а также в вещах и в сознании (\textit{привет, Иоанн Скот Эриугена})

Иоанн Росцелин (1051-1122)\\
А этот- \textbf{номиналист}. Общие понятия - только слова, и ничего более. Больше его работ не дошло, т.к. был еретиком.

Пьер Абеляр (1079-1142)\\
Тоже \textbf{номиналист}, но уже не такой, как \underline{Иоанн Росцелин}: общее - не только слово, но и понятие. Т.е. оно существует в человеческом сознании. Также ставил знание выше веры. В сочинении "Да и нет" привел противоречия в Библии. Также написал "Историю моих бедствий", хотя это уже не о философии.

Иоахим Флорский (Калабрийский) ака Джоаккино да Фьоре  (1130-1202)\\
Делил историю человечества на три эпохи: Бога-Отца, Бога-Сына и святого духа.

Аль-Кинди (800-877)\\
Жил в Ираке (город рождения не известен точно - Басра или Куфа. Умер в Багдаде.) Философ и естествоиспытатель. Первый арабский \textbf{аристотелист}. Уважал знание, делил его на три ступени: логику, естественные науки и философию. Подвергся гонениям на религиозной почве.

Аль-Фараби (870-950)\\
Родился в Казахстане, свалил в Дамаск. \textbf{Аристотелист}, но с оговорками. Мир, например, не вечен, а создан Аллахом. Религию считал низшей формой знания, облаченной в метафоры для упрощения понимания ее простым народом. А вот философия - это да, это высшая форма знания, для небыдла.

Ибн Сина ака Авиценна (980-1050)\\
Жил в Бухаре. \textbf{Аристотелист} и даже мир вечен. Бог - это мышление, мыслящее о самом себе. В мир это мышление не вмешивается. Общее у него существует в божественном разуме, вещах и понятиях (\textit{прям-таки Иоанн Скот Эриугена arab edition}). Медик (тільки не так пишеться).

Аль-Газали (1058-1111). Иран, Тус\\
Богослов. Аристотелистов обвинял в безбожии, отступлении от Корана, а главным способом познания считал интуицию. 

Ибн Рушд ака Аверроэс (1126-1198)\\ 
Из Кордовы. \textbf{Аристотелист} до мозга костей, продолжатель материалистической линии в трудах Аристотеля. Бог - это мышление, мыслящее о самом себе, как у \underline{Ибн Сины}. Душа у него неразрывно связана с телом: бессмертия нет (хотя и говорит о бессмертии человечества). Вещи не пришли извне, а всегда принадлежали миру (а он вечен). Религия и философия у него - по-разному сформулированная истина, как у \underline{Аль-Фараби}.

Джованни Фиданца ака Бонавентура (1221-1274)\\ 
Генерал ордена францисканцев. Идеи у него существуют, но высшее познание дано только Богу. Теология у него во главе всех наук. Познание у него имеет три формы: познание вещей, созданных Богом, познание собственной души, слияние с Богом. Вера, естественно, выше знания.

Альберт Великий  ака Альберт фон Больштедт ака Doctor Universalis (1206-1286)\\
По происхождению - немец, но учился в Падуе, потом ездил и преподавал. Вступил в Доминиканский орден. Считал, что необходимо знать греческую философию. Философию и богословие различал. У него две истины - разума и откровения, и вера важнее знания.

Фома Аквинский (1225-1274)\\
 Тоже вступил в Доминиканский орден. Ученик \underline{Альберта Великого}. Продолжал гнуть линию про две истины, но утверждал, что эти две истины отличаются только по форме. Важны и знания, и откровения. Но философия таки обслуживает веру. Утверждал, что есть недоказуемые и непостижимые для человеческого разума истины. Общее у него опять имеет троякое бытие (см. Иоанн Скот Эриугена). \textbf{Соц. реалист} - общество у него существует. В обществе неизбежно неравенство. Монархист - пусть король владеет телом, а церковь - душой. Его философия - \textbf{томизм} - в пропатченном виде и поныне юзается церьковью.

Сигер Брабантский (1235-1284)\\
Аверроист. Француз. Как еретик, подвергся гонениям. Душа смертна, человечество бессмертно, истина двойственна.
 
Роджер Бэкон (1214-1294)\\ 
Аверроист. Окончил Оксфорд, свалил  Париж. Тоже подвергся гонениям, лишен права преподавать, арестован, умер в тюрьме. Критик феодального строя. Философия, по его мнению, должна давать путь к познанию. Уважал математику и оптику. Считал скорость света конечной.

Иоанн Дунс Скот (1266-1308)\\
Окончил Оксфорд. Преподавал в Оксфорде, Париже, а умер в Кельне. Завершил теорию двух истин - теперь между ними нет ничего общего. Религия недоказуема, богословие не наука. Душа смертна, человечество бессмертно, истина двойственна. \textbf{Реалист} - общее есть, но науки движутся от общего к отдельному. Т.е. отдельное у  него - основа. \textbf{Сенсуалист}
 
Уильям Оккам (1285-1349)\\
Доказывал, что папство - временный институт. \textbf{Крайний номиналист} - познание у него интуитивное (т.е. чувственное), которое дает только отдельное. А дальше уже разум городит абстракции.

Франческо Петрарка (1304-1374)\\
Создатель литературного итальянского. Имел труды по истории.

Джованни Бокаччио (1313-1375)\\ 
Критиковал Средневековые порядки и духовенство в своей книге "Декамерон". 

Лоренцо Валла (1407-1457)\\
Указывал на необходимость критики источников, как античных, так и библейских. Разоблачил "Константинов дар".

Пьеро Помпанацци (1462-1525)\\
Довольно антирелигиозен - указывал, что три монотеистические религии - христианство, иудаизм, ислам - противоречат друг другу, а потому хотя бы две -  фейк. Написал "Трактат о трех обманщиках".

Никколо Макиавелли(1469-1527)\\
Политический деятель из Флоренции. Написал книгу "Государь". Мечтал о единой Италии. В книге "История Флоренции" есть намеки на объективность истории, но явно он такого не говорит.

Николай Кузанский ака Кребс (1404-1464)\\
Кардинал. Вселенная у него бесконечна, а Земля - не ее центр. Изучал отличие рассудка а разума.

Эразм Роттердамский (1467-1536)ъъ
Мечтал об освобождении Нидерландов от Испании. С севером прокатило. Написал "Похвальное слово глупости" - глупость у него там госпожа мира. Схоластику критиковал.

Томас Мор (1478-1535)\\
Лорд-канцлер Англии. Писал про Утопию. Понял, что беда - в частной собственности, из-за нее людей эксплуатируют. На острове Утопия же у него бы чуть ли не коммунизм. Из-за разногласий по поводу англиканской церкви был репрессирован - отрубили голову.

Ульрих фон Гуттен (1488-1523)\\
Написал "Письма темных людей". Настолько смачная пародия на схоластов, что сначала приняли за оригинал.

Томас Мюнцер (1490-1525)\\
В "Великую крестьянскую войну" был предводителем крестьян. Последователь \underline{Иоахима Флорского}, но ждать не хотел.

Франсуа Рабле (1490-1553)\\
Был монахом, но потом сбросил рясу. Написал сатирический роман "Гаргантюа и Пантагрюэль".

Пьер де ля Раме (1515-1575)\\
Ненавидел богословие, схоластику. Магистерскую защитил, разгромив Аристотеля, правда, с годами стал умереннее. Убит в Варфоломеевскую ночь как гугенот.

Мишель Монтень (1533-1592)\\
Был скептик - в том смысле, что требовал проверять все утверждения. Написал книгу "Опыты".

Жан Боден (1533-1592)\\Написал книгу "Метод легкого изучения истории". Выдвинул концепцию прогресса человечества.

Леонардо да Винчи (1533-1592)\\
Художник, изобретатель. Человек у него лишь первый из зверей, а душа не бессмертна. Ценил опыт, но и чистый эмпиризм без анализа отвергал.

Николай Коперник (1473-1543)\\
В книге "Об обращении небесных сфер" ввел гелиоцентрическую систему. Утверждал, что это лишь способ уточнить расчеты (правда, его модель была далеко не идеальна). Церковь это дело проморгала и не забанила вовремя. А когда спохватилась, то:

Джордано Бруно (1548-1600)\\
Горячий парень. Безбожник. Нет двух истин - есть только истина разума, а религия - ерундистика. Мир бесконечен, звезды - те же солнца, с планетами и разумной жизнью. Правда, и мировая душа у него была, но это не Бог никаким боком. Естественно, инквизиция такие мысли приветствовала горячо, даже пламенно. Бруно сидел, подвергался пыткам, но от теории не отрекся. В итоге его теория вместе с ним самим была отправлена в топку. \textit{Сволочь ты, инквизиция. Такого человека сгубила!}

Томас Гоббс (1588 - 1679)

Джон Локк (1632 - 1704)

Рене Декарт (1596 - 1650)

Бенедикт Спиноза (1632 - 1677)

Готфрид Лейбниц (1646 - 1716)

Иммануил Кант (1724 - 1804)\\
Персонаж игры Socrates Jones: Pro Philosopher.

Иоганн Фихте (1762 - 1814)

Фридрих Шеллинг (1175 - 1854)

Георг Гегель (1770 - 1831)

Огюст Конт (1798 - 1857)

Джон Стюарт Милль (1806 - 1873)\\
Персонаж игры Socrates Jones: Pro Philosopher.

Герберт Спенсер(1820 - 1903)

Рихард Авенариус (1843 - 1896)

Эрнст Мах (1838 - 1916)

Джордж Мур (1873 - 1958)

Бертрам Рассел (1872 - 1970)\\
Хороший обзорщик философов. Его книгу "История западной философии и её связи с политическими и социальными условиями от античности до наших дней" даже Семенов советует, хотя и не разделяет взглядов Рассела.

Людвиг Витгенштейн (1889 - 1951)

Карл Поппер (1902 - 1994)

\textbf{А теперь самые главные имена:}

Карл Маркс (1818 - 1883)

Фридрих Энгельс (1820 - 1895)

Владимир Ленин (1870 - 1924)