\section{Аппендиксъ 2: Философский бестиариум <WIP>}
{\scriptsize
\begin{tabular}{|c|c|c|}
\hline
\thead{Кто\\Когда и Где}&\thead{Философия}&\thead{Кроме философии}

\phylentry{Фалес}
{($\approx$624 д.н.э - $\approx$546 д.н.э.)}
{Милет, Малая Азия\\\
(Гонял по заграницам)}
{-Мир вечен\\\
-Первовещество-вода}
{-Смотался в Египет, там ботал преднауку\\\
-Первый доказал теорему\\\
-Предсказывал солнечные затмения}{}

\phylentry{Анаксимандр}
{($\approx$619 д.н.э - $\approx$546 д.н.э.)}
{Милет, Малая Азия}
{-Мир вечен\\\
-Первовещество-апейрон\\\
-Мир-цилиндр\\\
-Человек-потомок морских животных}
{-Смотался в Египет, там ботал преднауку\\\
-Первый доказал теорему\\\
-Предсказывал солнечные затмения}{}


\phylentry{Анаксимен}
{($\approx$585 д.н.э - $\approx$525 д.н.э.)}
{Милет, Малая Азия}
{-Мир вечен\\\
-Первовещество-воздух\\\
-Мир-диск}
{ВОИДЪ}{}

\phylentry{Гераклит}
{($\approx$540 д.н.э - $\approx$480 д.н.э.)}
{Эфес, Малая Азия}
{-Мир вечен\\\
-Первовещество-огонь\\\
-Диалектика: мир-вечный процесс\\\
-Его двигатель - противоположности и противоречия}
{ВОИДЪ}{}

\\\hline
\end{tabular}
}

Гераклит ($\approx$540 д.н.э - $\approx$480 д.н.э.)\\
Объединяется вместе с предыдущими 3-мя в "Ионийскую школу". Как и милетцы, считал мир вечным, в котором нет ничего вечного. Тоже вводил первовещество -  \textbf{огонь}. Но интересен он также тем, что видел движущую силу в противоречивости этого мира (\textit{по типу "нож-это оружие, но в руках хирурга он спасает жизни и т.д."}). Получается, мир состоит не из набора вещей, а из процессов. Т.о. Гераклит - создатель \textbf{диалектики}. 

Пифагор ($\approx$570 д.н.э - $\approx$487 д.н.э.)\\
Родился на острове Самосе, но жил и работал в южной Италии. Основатель пифагорейской школы. Доказал теорему имени себя. Выделял три области исследований - религию, этику и науку. Пифагор и пифагорейцы чуть ли не обожествляли числа: все явления - это числа. Мир описывали как центральный огонь (\textit{не Солнце}), вокруг которого вращаются десять сфер: пять планет, Земля, Солнце, Луна, Млечный путь и Антиземля (чтобы всего было десять). Делили все на десять пар противоположных понятий, например: "четное-нечетное", " мужское-женское", "четырехугольное-многостороннее","свет-тьма" и т.д.

Ксенофан ($\approx$580 д.н.э - $\approx$490 д.н.э.)\\ 
Относится к Элейской школе. Бродячий певец родом из Малой Азии, обошел всю Грецию. \textbf{Пантеист}-отождествлял Бога и природу, бытие. Мифологию Древней Греции не признавал, считал выдумкой людей.

Парменид ($\approx$546 д.н.э - $\approx$470 д.н.э.)\\
Элейская школа. Что такое бытие? Это - то, что есть. А небытие? То, чего нет. Раз так, то бытие вечно, иначе откуда оно пришло и куда уйдет? В небытие? Так некуда, получается. А раз так, то и в мире ничто не появляется и не исчезает и не меняется. Итого есть только неподвижное и единое (иначе два бытия отграничены небытием) бытие, а наши чувства нас обманывают. А это, кстати, уже важная мысль - Парменид поделил познание на чувственное и умственное. 

Зенон Элейский ($\approx$490 д.н.э - $\approx$430 д.н.э.)\\ 
Элейская школа. Ученик Парменида, доказывал своего учителя \textbf{апориями} - парадоксами, из которых следовала невозможность движения. До нас дошло не все. Например: "Летящая стрела неподвижна, так как в каждый момент времени она занимает равное себе положение, то есть покоится; поскольку она покоится в каждый момент времени, то она покоится во все моменты времени, то есть не существует момента времени, в котором стрела совершает движение." Или про Ахиллеса, черепаху и сходимость рядов.

Мелисс ($\approx$485 д.н.э - $\approx$425 д.н.э.)\\
Элейская школа. Его важнейшее положение - бесконечность мира не только во времени, но и в пространстве. Опять же, явления - видимости, а есть только сущности.

Эмпедокл ($\approx$492 д.н.э - $\approx$430 д.н.э.)\\
Вводил четыре элемента: огонь, воду, землю и воздух. Они вечны, а все сущее - результат смешения этих стихий под действием двух сил - \textbf{дружбы и вражды}. \textbf{Редукционист} - сводит вещи к смеси старых качеств. У него было много интересных мыслей - про конечность скорости света, про естественный отбор (правда, это было скорее соединение нежизнеспособных органов в монстров, которые в итоге стали гармоничными и способными к размножению). А ощущение он понимал так: частицы стихий стряхиваются и попадают на наши органы чувств.

Анаксагор ($\approx$496 д.н.э - $\approx$428 д.н.э.)\\
У него есть семена мира, их, в отличие от Эмпедокла, бесчисленное множество. Движущая сила, объединяющая их - \textbf{ум (нус)}. Т.е. тела - просто смешение семян мира.

Демокрит ($\approx$460 д.н.э - $\approx$360 д.н.э.)\\
Ярчайший представитель \textbf{атомистического материализма}. Ученик Левкиппа, который и предполагается как создатель сего направления \textit{(но увы и ах, данные неточные)}.
Его взгляды неясны до конца - возможно, его точка зрения менялась в процессе. Основные же положения: есть пустота, небытие. В ней есть мельчайшие неделимые частицы - \textbf{атомы}\textit{(Неделимые? Резерфорд, Кюри и Майтнер под столом.)}. Эти атомы обладают весом, формой и незримы. Из них-то все и состоит. Да, и чувства нас опять-таки обманывают. Мир не такой, как в сознании. \textit{Говорят, на старости лет даже глаза себе выколол, чтоб постигать мир исключительно разумом}. И важнейшее его открытие - \textbf{детерминизм}. Атомы подчиняются всеобщим законам, а значит, все имеет причину и предопределено. Необходимость тоже всеобщая. 

Протагор ($\approx$486 д.н.э - $\approx$411 д.н.э.)\\
Считал, что человек - мера всех вещей. А раз так, то и у каждого своя правда. Истина есть то, что приносит пользу. А полезность субъективна.

Горгий ($\approx$480 д.н.э - $\approx$380 д.н.э.)\\
На протяжении жизни жил в Италии, Афинах, умер в Ларисе. Основные мысли: ничто не существует; если же и существует, то не познаваемо; а если и познаваемо, то языком непередаваемо.

Сократ ($\approx$469 д.н.э - $\approx$399 д.н.э.)\\
Книг не писал, вел беседы (называл методику \textbf{майевтикой} - искусством повивальной бабки): вначале включал иронию, и когда оппонент понимал, насколько безблагодатная у него точка зрения, Сократ наводящими вопросами помогал ему родить мысль. Был обвинен в развращении молодежи и богохульстве. Приговорен к казни и выпил яду. Что же касается его философских воззрений, то он считал, что добродетель проистекает из знания: "никто не желает зла по своей воле - зло от незнания". Спорами восстанавливал авторитет истины, опущенной софистами. Тремя главными добродетелями считал мужество, справедливость и умеренность. Говорил, что круг потребностей нужно ограничить, чтобы не зависеть от других людей (эту мысль развили \textbf{киники}).

Платон (Аристокл) ($\approx$427 д.н.э - $\approx$378 д.н.э.)\\
Афинянин, но после смерти Сократа покинул Афины. Создатель философской школы - Академии. Родоначальник эстетики. Придерживался \textbf{объективного идеализма}. Он отделил слово от понятия: первое материально, второе идеально. Пришел к идее, что есть мир вещей и мир идей, задавшись вопросом о том, существует ли общее в реальном мире. Идеи вечны, а вещи приходят и уходят. У человека есть душа, она вечна и пребывает в мире идей, если не находится в теле. Высшая идея - благо. Душа вспоминает идеи, которые она видела в том мире - отсюда у нас есть какие-то понятия (например, о том же благе и красоте). У него есть и материя, из которой состоят вещи. И еще у него есть Бог, который, руководствуясь идеями, творит мир из материи. Душу человека делил на три: ощущающую, пылкую и разумную. Первые две сравнивал с конями, а третью - с извозчиком. У него есть классификация государственных строев: власть одного (царь/тиран), власть группы (аристократия/олигархия), власть большинства (демократия). Описывал трехклассовое идеальное общество - правители-философы, стражи, крестьяне-ремесленники.

Аристотель ($\approx$384 д.н.э - $\approx$322 д.н.э.)\\
Родился в Ставире, жил в Афинах. Путешествовал по Греции. Воспитатель Александра Македонского. Прошел Академию, но впоследствии основал свою школу - Ликей (Лицей). Бежал из Афин после смерти Александра Македонского. Геоцентрист, считал, что есть 55 небесных сфер. На Земле все состоит из огня/воды/земли/воздуха, и так до орбиты Луны, а все,что выше - эфир. У Аристотеля было понятие материи. Из нее состоят все вещи. Но материя сама по себе пассивна: чтобы образовать вещь, она должна соединиться с формой. Бог - высшая форма.Разделял чувственное познание и мышление. У него было два разума - пассивный человеческий и активный божественный. В человеческом есть общее знание, которое извлекается миром и активным разумом. Создал формальную логику.

Антисфем (435 д.н.э.-370 д.н.э.)\\
Создатель направления киников.

Диоген (390 д.н.э.-323 д.н.э.)\\ 
Философ-киник, мизантроп, онанист и просто хороший человек.

Эпикур (341 д.н.э.-279 д.н.э.)\\
Философию делил на логику (учение о познании), физику (учение о природе) и этику (учение об общественной жизни). \textbf{Атомист}, но отказался от детерминизма. Атомы без случайности просто бы падали вниз, не образуя вещей. Считал благом наслаждение, а злом страдание, при этом не сводился к гедонизму: есть потребности естественные необходимые, естественные не необходимые и неестественные не необходимые. Наивысшее наслаждение - беседа с мудрецом. Добродетели у него - апатия, бесстрастие, невозмутимость.

Тит Лукреций Кар (99 д.н.э.-55 д.н.э.)\\
Последователь Эпикура. Написал книгу "О природе вещей", в которой и расписал его взгляды (возможно, с добавлением собственных).

Зенон Китийский (334 д.н.э.-262 д.н.э.)\\ 
Создатель направления стоиков. Был купцом с Кипра, корабль разбился, стол философом в Афинах. 

Крисипп (280 д.н.э.- 205 д.н.э.)\\
Еще один стоик.

Пиррон (365 д.н.э.-275 д.н.э.)\\ 
Скептик с Пелопоннеса.

Тимон (320 д.н.э.-230 д.н.э.)\\
Те же идеи, что и у Пиррона.

Цицерон (106 д.н.э. - 46 д.н.э.)

Луций Анний Сенека  (6 д.н.э.-65 н.э.)\\
Воспитатель Нерона. Крупный администратор. В 41-49 годах находился в ссылке. Как и Сократ, покончил с собой по приговору суда. Стоик. Не соглашался с Сократом в том, что знание дает добродетель. Не считал зазорным быть богатым, но требовал использовать богатство с умом. Дух судьбы чуть ли не обожествлял.

Эпиктет (55 н.э.- 130 н.э.)\\
Еще один стоик. Был рабом, получил вольную. Странствовал по Албании и вел беседы. Раб у него может быть свободным, а господин - рабом. Главное счастье - быть невозмутимым и свободным.

Марк Аврелий Антонин (121 н.э.-180 н.э.)\\
Император-стоик. Призывал к смирению.

Тертуллиан (155 н.э.-225 н.э.)\\
Карфагенец. Критик античной философии - язычники же все. Выразил концепцию Троицы. Верует, ибо абсурдно. Создал \textbf{патристику} - ХГМ-направление в философии.

Климент Александрийский (150 н.э.-220 н.э.)\\
Считал, что христианство можно доказать, используя достижения античной философии.

Ориген (185 н.э.-254 н.э.)\\
Те же взгляды, что и у предыдущего. Еретик, ибо ставил Бога-Отца выше Сына.

Аммоний Саккас (175 - 242)\\
Александриец. Основатель неоплатонизма.

Плотин (206 - 275)\\
Ученик предыдущего. У него есть понятие единого первоначала. Из него получаются божественный ум, божественная душа и природа.

Порфирий (204 - 301)

Ямвлих (245 - 330)\\
Какое-то богословие и только

Прокл (410 - 485)

Августин Аврелий Блаженный (354 - 430)
Создал первую философию истории.Она конечто, далека от реальности, но все же: Бог наметил план истории, и история подстраивается под него - \textbf{провиденционализм}. Есть Град земной - в нем живут по плоти, и Град божий - там живут по духовным ценностям (\textit{с христанутостью это не скоррелировано!}). На Страшном Суде их пихают в Ад и Рай соответственно. Государство он приравнивал к банде. Концепция у него унитарная - человечество движется как единое целое. Но прогресса нет, а будет только конец света.

Северин Боэций (480-524)\\
Был приговорен к казни за участие в заговоре, в тюрьме написал книгу "Утешение философией", в которой занимал позиции стоиков. Утверждал, что Бог знает будущее, но это не отменяет свободной воли индивида. Переводил труды.

Аврелий Кассиодор ($\approx$485-$\approx$585)\\
Создатель концепции "Семи свободных искусств". Они делись на две ступени - \textbf{тривиум} и \textbf{квадрилиум}.

Иоанн Скот Эриугена (810-877)\\
Ирландец. Жил и работал при дворе Карла Великого. С примесью  богословия, но все-таки философ. Наследник патристики. Ставил знание выше веры. То, что все знание содержится в Библии, означает, что его нужно уметь подчерпнуть оттуда. Считал, что общее существует - был \textbf{реалистом}. Считал, что есть три формы существования общего: в божественном разуме, в вещах, в голове человека.

Ансельм Кентерберийский Д'Аоста. (1033-1109)\\
Глава английской церкви. Веру ставил выше знания: да, без знания никуда, но оно служит вере. Придумал \textbf{онтологическое доказательство} существования Бога: "Бог - совершенное существо. Но как может совершенное существо не существовать" (\textit{софист долбаный!}). \textbf{Реалист} - считал, что общее существует  - это план, по которому Бог создавал этот мир; а также в вещах и в сознании (\textit{привет, Иоанн Скот Эриугена})

Иоанн Росцелин (1051-1122)\\
А этот- \textbf{номиналист}. Общие понятия - только слова, и ничего более. Больше его работ не дошло, т.к. был еретиком.

Пьер Абеляр (1079-1142)\\
Тоже \textbf{номиналист}, но уже не такой, как \underline{Иоанн Росцелин}: общее - не только слово, но и понятие. Т.е. оно существует в человеческом сознании. Также ставил знание выше веры. В сочинении "Да и нет" привел противоречия в Библии. Также написал "Историю моих бедствий", хотя это уже не о философии.

Иоахим Флорский (Калабрийский) ака Джоаккино да Фьоре  (1130-1202)\\
Делил историю человечества на три эпохи: Бога-Отца, Бога-Сына и святого духа.

Аль-Кинди (800-877)\\
Жил в Ираке (город рождения не известен точно - Басра или Куфа. Умер в Багдаде.) Философ и естествоиспытатель. Первый арабский \textbf{аристотелист}. Уважал знание, делил его на три ступени: логику, естественные науки и философию. Подвергся гонениям на религиозной почве.

Аль-Фараби (870-950)\\
Родился в Казахстане, свалил в Дамаск. \textbf{Аристотелист}, но с оговорками. Мир, например, не вечен, а создан Аллахом. Религию считал низшей формой знания, облаченной в метафоры для упрощения понимания ее простым народом. А вот философия - это да, это высшая форма знания, для небыдла.

Ибн Сина ака Авиценна (980-1050)\\
Жил в Бухаре. \textbf{Аристотелист} и даже мир вечен. Бог - это мышление, мыслящее о самом себе. В мир это мышление не вмешивается. Общее у него существует в божественном разуме, вещах и понятиях (\textit{прям-таки Иоанн Скот Эриугена arab edition}). Медик (тільки не так пишеться).

Аль-Газали (1058-1111). Иран, Тус\\
Богослов. Аристотелистов обвинял в безбожии, отступлении от Корана, а главным способом познания считал интуицию. 

Ибн Рушд ака Аверроэс (1126-1198)\\ 
Из Кордовы. \textbf{Аристотелист} до мозга костей, продолжатель материалистической линии в трудах Аристотеля. Бог - это мышление, мыслящее о самом себе, как у \underline{Ибн Сины}. Душа у него неразрывно связана с телом: бессмертия нет (хотя и говорит о бессмертии человечества). Вещи не пришли извне, а всегда принадлежали миру (а он вечен). Религия и философия у него - по-разному сформулированная истина, как у \underline{Аль-Фараби}.

Джованни Фиданца ака Бонавентура (1221-1274)\\ 
Генерал ордена францисканцев. Идеи у него существуют, но высшее познание дано только Богу. Теология у него во главе всех наук. Познание у него имеет три формы: познание вещей, созданных Богом, познание собственной души, слияние с Богом. Вера, естественно, выше знания.

Альберт Великий  ака Альберт фон Больштедт ака Doctor Universalis (1206-1286)\\
По происхождению - немец, но учился в Падуе, потом ездил и преподавал. Вступил в Доминиканский орден. Считал, что необходимо знать греческую философию. Философию и богословие различал. У него две истины - разума и откровения, и вера важнее знания.

Фома Аквинский (1225-1274)\\
 Тоже вступил в Доминиканский орден. Ученик \underline{Альберта Великого}. Продолжал гнуть линию про две истины, но утверждал, что эти две истины отличаются только по форме. Важны и знания, и откровения. Но философия таки обслуживает веру. Утверждал, что есть недоказуемые и непостижимые для человеческого разума истины. Общее у него опять имеет троякое бытие (см. Иоанн Скот Эриугена). \textbf{Соц. реалист} - общество у него существует. В обществе неизбежно неравенство. Монархист - пусть король владеет телом, а церковь - душой. Его философия - \textbf{томизм} - в пропатченном виде и поныне юзается церьковью.

Сигер Брабантский (1235-1284)\\
Аверроист. Француз. Как еретик, подвергся гонениям. Душа смертна, человечество бессмертно, истина двойственна.
 
Роджер Бэкон (1214-1294)\\ 
Аверроист. Окончил Оксфорд, свалил  Париж. Тоже подвергся гонениям, лишен права преподавать, арестован, умер в тюрьме. Критик феодального строя. Философия, по его мнению, должна давать путь к познанию. Уважал математику и оптику. Считал скорость света конечной.

Иоанн Дунс Скот (1266-1308)\\
Окончил Оксфорд. Преподавал в Оксфорде, Париже, а умер в Кельне. Завершил теорию двух истин - теперь между ними нет ничего общего. Религия недоказуема, богословие не наука. Душа смертна, человечество бессмертно, истина двойственна. \textbf{Реалист} - общее есть, но науки движутся от общего к отдельному. Т.е. отдельное у  него - основа. \textbf{Сенсуалист}
 
Уильям Оккам (1285-1349)\\
Доказывал, что папство - временный институт. \textbf{Крайний номиналист} - познание у него интуитивное (т.е. чувственное), которое дает только отдельное. А дальше уже разум городит абстракции.

Франческо Петрарка (1304-1374)\\
Создатель литературного итальянского. Имел труды по истории.

Джованни Бокаччио (1313-1375)\\ 
Критиковал Средневековые порядки и духовенство в своей книге "Декамерон". 

Лоренцо Валла (1407-1457)\\
Указывал на необходимость критики источников, как античных, так и библейских. Разоблачил "Константинов дар".

Пьеро Помпанацци (1462-1525)\\
Довольно антирелигиозен - указывал, что три монотеистические религии - христианство, иудаизм, ислам - противоречат друг другу, а потому хотя бы две -  фейк. Написал "Трактат о трех обманщиках".

Никколо Макиавелли(1469-1527)\\
Политический деятель из Флоренции. Написал книгу "Государь". Мечтал о единой Италии. В книге "История Флоренции" есть намеки на объективность истории, но явно он такого не говорит.

Николай Кузанский ака Кребс (1404-1464)\\
Кардинал. Вселенная у него бесконечна, а Земля - не ее центр. Изучал отличие рассудка а разума.

Эразм Роттердамский (1467-1536)ъъ
Мечтал об освобождении Нидерландов от Испании. С севером прокатило. Написал "Похвальное слово глупости" - глупость у него там госпожа мира. Схоластику критиковал.

Томас Мор (1478-1535)\\
Лорд-канцлер Англии. Писал про Утопию. Понял, что беда - в частной собственности, из-за нее людей эксплуатируют. На острове Утопия же у него бы чуть ли не коммунизм. Из-за разногласий по поводу англиканской церкви был репрессирован - отрубили голову.

Ульрих фон Гуттен (1488-1523)\\
Написал "Письма темных людей". Настолько смачная пародия на схоластов, что сначала приняли за оригинал.

Томас Мюнцер (1490-1525)\\
В "Великую крестьянскую войну" был предводителем крестьян. Последователь \underline{Иоахима Флорского}, но ждать не хотел.

Франсуа Рабле (1490-1553)\\
Был монахом, но потом сбросил рясу. Написал сатирический роман "Гаргантюа и Пантагрюэль".

Пьер де ля Раме (1515-1575)\\
Ненавидел богословие, схоластику. Магистерскую защитил, разгромив Аристотеля, правда, с годами стал умереннее. Убит в Варфоломеевскую ночь как гугенот.

Мишель Монтень (1533-1592)\\
Был скептик - в том смысле, что требовал проверять все утверждения. Написал книгу "Опыты".

Жан Боден (1533-1592)\\Написал книгу "Метод легкого изучения истории". Выдвинул концепцию прогресса человечества.

Леонардо да Винчи (1533-1592)\\
Художник, изобретатель. Человек у него лишь первый из зверей, а душа не бессмертна. Ценил опыт, но и чистый эмпиризм без анализа отвергал.

Николай Коперник (1473-1543)\\
В книге "Об обращении небесных сфер" ввел гелиоцентрическую систему. Утверждал, что это лишь способ уточнить расчеты (правда, его модель была далеко не идеальна). Церковь это дело проморгала и не забанила вовремя. А когда спохватилась, то:

Джордано Бруно (1548-1600)\\
Горячий парень. Безбожник. Нет двух истин - есть только истина разума, а религия - ерундистика. Мир бесконечен, звезды - те же солнца, с планетами и разумной жизнью. Правда, и мировая душа у него была, но это не Бог никаким боком. Естественно, инквизиция такие мысли приветствовала горячо, даже пламенно. Бруно сидел, подвергался пыткам, но от теории не отрекся. В итоге его теория вместе с ним самим была отправлена в топку. \textit{Сволочь ты, инквизиция. Такого человека сгубила!}

Томас Гоббс (1588 - 1679)

Джон Локк (1632 - 1704)

Рене Декарт (1596 - 1650)

Бенедикт Спиноза (1632 - 1677)

Готфрид Лейбниц (1646 - 1716)

Иммануил Кант (1724 - 1804)\\
Персонаж игры Socrates Jones: Pro Philosopher.

Иоганн Фихте (1762 - 1814)

Фридрих Шеллинг (1175 - 1854)

Георг Гегель (1770 - 1831)

Огюст Конт (1798 - 1857)

Джон Стюарт Милль (1806 - 1873)\\
Персонаж игры Socrates Jones: Pro Philosopher.

Герберт Спенсер(1820 - 1903)

Рихард Авенариус (1843 - 1896)

Эрнст Мах (1838 - 1916)

Джордж Мур (1873 - 1958)

Бертрам Рассел (1872 - 1970)\\
Хороший обзорщик философов. Его книгу "История западной философии и её связи с политическими и социальными условиями от античности до наших дней" даже Семенов советует, хотя и не разделяет взглядов Рассела.

Людвиг Витгенштейн (1889 - 1951)

Карл Поппер (1902 - 1994)

\textbf{А теперь самые главные имена:}

Карл Маркс (1818 - 1883)

Фридрих Энгельс (1820 - 1895)

Владимир Ленин (1870 - 1924)