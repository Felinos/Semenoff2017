\section{Аппендиксъ 1: Словарь терминов}
\hspace{15pt}\textbf{Агностицизм} - философское течение, в котором утверждается невозможность дать ответ на основной вопрос (\textit{про мир и сознание}), по причине того, что невозможно познать что-либо за пределами нашего сознания.

\textbf{Акциденция} - это бытие производное, имеющее началом что-то отличное от него самого.

\textbf{Атеизм} -  система взглядов, отрицающих существование богов. \textit{"Атеизм является отрицанием бога и утверждает бытие человека именно посредством этого отрицания" (Маркс)}.

\textbf{Восприятие} - целостный образ предмета внешнего мира.

\textbf{Гносеология} - теория процесса познания истины, теория познания.

\textbf{Деизм} -  религиозно-философское направление, признающее существование Бога и сотворение им мира, но отрицающее большинство сверхъестественных и мистических явлений, Божественное откровение и религиозный догматизм. Большинство деистов полагают, что Бог после сотворения мира не вмешивается в течение событий (как некий великий часовщик, который сделал часы и больше не вмешивается в их ход).

\textbf{Детерминизм} - учение о взаимосвязи и взаимной определенности всех явлений и процессов, доктрина о всеобщей причинности.

\textbf{Диалектика} - многозначненько:
\begin{enumerate}
\item У Гераклита - учение о том, что мир - непрерывный, вечный процесс.
\item В тривиуме - это одна из дисциплин, под которой понималась философия.
\item Также понимается как предельно общий метод мышления.
\end{enumerate}

\textbf{Дуализм} - это философское течение с двумя субстанциями.

\textbf{Заблужение} - противоположность истины, несовпадение между миром и сознанием \textit{Ложь - умышленное введение в заблуждение, в этом отличие.}

\textbf{Идеализм} - философское направление, в котором сознание первично, мир вторичен.

\textbf{Объективный, идеализм} - течение идеализма, в котором присутствует объективный абсолютный дух.

\textbf{Субъективный, идеализм} - течение идеализма, в котором весь мир существует исключительно в сознании субъекта.

\textbf{Социоконструктивный, идеализм} - течение идеализма, в котором роль абсолютного духа играет общественное сознание.

\textbf{Идеальное} - принадлежащее сознанию, нематериальное, духовное.

\textbf{Истина} - это то, что согласуется с действительностью; соответствие между миром и сознанием.

\textbf{Логика} - это наука о мышлении, в центре внимания которой - его правильность или истинность.

\textbf{Материальное} - принадлежащее миру.

\textbf{Материализм} - философское направление, в котором мир первичен, а сознание вторично.

\textbf{Метод познания} - способ достижения истины.

\textbf{Метод мышления} - способ достижения истины.

\textbf{Мышление} - ступень человеческого познания, подвластная сознанию.

\textbf{Натурфилософия} - разновидность философии, которая стремится дать цельную картину мира. 

\textbf{Неотомизм} - доработанная философия Фомы Аквинского. \textit{Попытка соединить томизм м новыми философскими идеями. Рабочая философия католической церкви.}

\textbf{Номинализм} - философское течение, утверждающее что общее или не существует, или существующее исключительно в голове человека.

\textbf{Общее} - принцип бытия всех единичных вещей, явлений, процессов. \textit{Что, звучит криво? FIXME, а то меня уже заколебало}.

\textbf{Объект} - то, что познается, предмет или явление.

\textbf{Объективное} - то, что не зависит от субъекта.

\textbf{Онтология} - учение о сущем; учение о бытии как таковом; \textit{раздел философии, изучающий фундаментальные принципы бытия, его наиболее общие сущности и категории, структуру и закономерности}.

\textbf{Отдельное (единичное)} - конкретные вещи/события.

\textbf{Ощущения} -  начальная фаза чувственного познания, в которой есть только разрозненные данные от органов чувств.

\textbf{Пантеизм} -  религиозное и философское учение, отождествляющее Бога и мир.

\textbf{Позитивизм} - философское направление, отрицающее возможность познания закономерных связей и отношений действительности и ограничивающее роль науки описанием фактов, явлений; философское учение и направление в методологии науки, определяющее единственным источником истинного, действительного знания эмпирические исследования и отрицающее познавательную ценность философского исследования. \textit{Курятник, БЛДЖАД, хотя с точки зрения формальной логики у них все ОК}.

\textbf{Понятие} - элементарная составляющая мышления.

\textbf{Представление} - образ ранее воспринятого предмета, который может быть воспроизведен в сознании субъекта в отсутствии этого предмета; последняя стадия чувственного познания.

\textbf{Разум} - мышление как объективный процесс.

\textbf{Рассудок} - мышления как субъективная деятельность.

\textbf{Реализм} - течение философии, в котором общее существует как нечеловеческое духовное. 

\textbf{Сенсуализм} - течение философии, в котором чувственное познание объявляется единственным источником знаний.

\textbf{Социальная философия} - предельно общий вгляд на общество.

\textbf{Ступени познания} - чувственное познание и мышление.

\textbf{Субстанция} - это бытие, существование которого не обусловлено никаким другим бытием, находящимся за его пределами.

\textbf{Схоластика} - систематическая европейская средневековая философия, представляющая собой синтез христианского (католического) богословия и логики Аристотеля.

\textbf{Томизм} - философия Фомы Аквинского, христианский аристотелизм. \textit{Учение о способах постижения догматов христианской веры}.

\textbf{Фатализм} -  вера в предопределенность бытия; мировоззрение, в основе которого убежденность в неизбежности событий, которые уже запечатлены наперед и лишь "проявляются" как изначально заложенные свойства данного пространства.

\textbf{Феноменализм} = агностицизм.

\textbf{Чувственное познание} - способ познания, который осуществляется органами чувств и человеку неподвластен.

\textbf{Эпистемология} = гносеология.